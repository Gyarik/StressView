\documentclass[10pt]{article}

\usepackage{svg, hyperref}
\hypersetup{hidelinks}

\title{StressView}
\author{Alessio Turetta, 2008069}
\date{}

\renewcommand{\contentsname}{Indice}
\renewcommand{\figurename}{Figura}

\begin{document}

\maketitle
\tableofcontents
\newpage

\section{Introduzione}
StressView è un'applicazione che permette la simulazione di
performance su un generico componente per PC.
I tipi di componente a disposizione sono tre dei più importanti, a scelta tra
Processore (CPU), Scheda Grafica (GPU) e Memoria (RAM).
Oltre ad offrire un grafico di andamento della performance, il programma segnala
potenziali e soprattutto gravi perdite rispetto all'andamento normale,
le quali si presentano diversamente a seconda del componente
e sulla base della temperatura massima inserita da utente (la quale varia
durante la simulazione stessa), allo scopo di simulare il cosiddetto "thermal
throttling", cioè la riduzione automatica di potenza per evitare il danneggiamento
dei componenti in caso di temperature eccessive.
\\\\
Ho scelto questo progetto perché mi è sembrato un tema adatto per la simulazione
di sensori e l'utilizzo di polimorfismo.

\section{Modello logico}
Il modello logico si basa principalmente sulla gestione di sensori come gerarchia
a partire da un sensore generico che contiene maggior parte dei metodi base
per ottenere, aggiungere e togliere i dati della simulazione.
I sensori sono poi immagazzinati in una classe container che prevede l'aggiunta,
la rimozione e la modifica dei sensori stessi, nonché la funzione di ricerca per la view.

\texttt{}
\begin{figure}[h]
\includesvg[width=\textwidth,height=\textheight,keepaspectratio]{img/Model.svg}
\caption{UML Modello Logico (immagine \texttt{.svg})}
\end{figure}
\newpage
In particolare (Figura 1) la classe base polimorfa, definita \texttt{GenericSensor},
rappresenta i campi, come il nome, descrizione, massimi, e dati della simulazione,
e i metodi comuni ai sensori concreti:
\begin{itemize}
    \item \texttt{CPUSensor}
    \item \texttt{GPUSensor}
    \item \texttt{RAMSensor}
\end{itemize}
Ognuno rispettivamente simula sensori dei tre componenti descritti.
I dati sono accessibili dal vettore, impostato a \texttt{protected} per
abilitare l'accesso esclusivamente alle classi figlie (quindi alle implementazioni
della classe astratta), contenente un tipo di dato personalizzato, definito nella
classe \texttt{PointInfo}, la quale contiene tutti i dati necessari per definire
un singolo punto della simulazione, di cui il campo \texttt{m\_bad} serve per indicare se
il dato è considerato un peggioramento in performance rispetto all'andamento
della simulazione. I sensori astratti sono salvati nella classe \texttt{SensorContainer},
per poterli gestire come una lista maneggiabile dalla view (es. aggiunta e rimozione pulsanti).
Infine ho utilizzato il design pattern Visitor per determinare il tipo dinamico
di un qualsiasi sensore, tramite le classi \texttt{GenericVisitor} e \texttt{Visitor},
allo scopo di ottenere le diverse unità di misura per il grafico e mostrare le icone
corrette a seconda del sensore.

\section{Polimorfismo}
Grazie al design pattern Visitor ho implementato il polimorfismo non banale
per la gerarchia \texttt{GenericSensor}, di fatto mi permette di trattare qualsiasi
sensore come tipo statico \texttt{GenericSensor*}, pur sempre mantenendo le differenze
tra le implementazioni, utile soprattutto nella view.
Il Visitor viene principalmente utilizzato per identificare i nomi dei diversi
componenti, le loro unità di misura nella simulazione, e impostare alcune grafiche.
Riguardo alla simulazione, ogni sensore agisce sui dati in maniera diversa:
\begin{itemize}
    \item \texttt{CPUSensor} - Se la temperatura supera una certa soglia (impostata nel model),
    vi è un'alta probabilità che la velocità di clock (in MHz) cali eccessivamente.
    \item \texttt{GPUSensor} - Oltre alla soglia di temperatura, viene controllato
    se il dato predente è segnato come "bad", cercando quindi di mantenere la performance
    peggiore.
    \item \texttt{RAMSensor} - La semantica cambia completamente essendo che il sensore osserva
    la performance in funzione della memoria utilizzata (in MiB), quindi normalmente
    l'utilizzo rimane basso, e la "perdita" si avrà quando la memoria non è più capace
    di gestire i dati alla velocità ottimale, causando un riempimento temporaneo.
\end{itemize}

\section{Persistenza dei dati}
Per la persistenza dei dati viene utilizzato il formato strutturato JSON, che permette
di salvare una lista di sensori e i loro dati simulati (se presenti), con
identificazione delle classi tramite l'attributo "component"; tutti gli altri
dati sono salvati con una banale associazione key-value che va a rispecchiare i
campi delle classi principali. Assieme alla relazione è presente un file di esempio
(\texttt{example.json}) che si può caricare nell'applicazione tramite la sezione del menù "File".

\section{Funzionalità}
Le funzionalità del programma sono le seguenti:
\begin{itemize}
    \item \textbf{Widget Informazioni Sensore}
    \begin{itemize}
        \item Stampa a schermo le informazioni base di un sensore
        \item Crea una nuova simulazione, svuota la simulazione, e
        permette la modifica delle informazioni tramite pulsanti
    \end{itemize}
    \item \textbf{Widget Lista Sensori}
    \begin{itemize}
        \item Mostra il container di sensori, creati e rimossi dinamicamente
        nella view e mostrati come pulsanti, con aggiunta di icone per mostrare
        il componente di un sensore
        \item Prevede tre pulsanti per aggiungere sensori, eliminare il sensore selezionato
        o eliminarli tutti
        \item Permette la ricerca di sensori per nome con una semplice barra di ricerca
    \end{itemize}
    \item \textbf{Widget Grafico Simulazione}
    \begin{itemize}
        \item Converte la simulazione nel model in un grafico, in cui viene
        sottolineato ogni punto in cui vi è una caduta di performance, tramite
        colori diversi ed etichette.
        \item La view contiene una funzione per eliminare completamente il grafico,
        utile nel caso di reset e/o cancellazione dei sensori
    \end{itemize}
    \item \textbf{Menu bar}
    \begin{itemize}
        \item Contiene i tasti per salvare e caricare file \texttt{.json}
        \item Include le rispettive shortcut da tastiera (\texttt{CTRL+s} e \texttt{CTRL+o})
    \end{itemize}
\end{itemize}

\newpage

\section{Riepilogo ore}
\begin{center}
\begin{tabular}{|c c c|}
    \hline
    \textbf{Attività} & \textbf{Ore Previste} & \textbf{Ore Effettive} \\
    \hline
    \multicolumn{1}{|l}{Studio e progettazione} & 5 & 7 \\
    \multicolumn{1}{|l}{Sviluppo del codice del modello} & 10 & 15 \\
    \multicolumn{1}{|l}{Studio del framework Qt} & 10 & 8 \\
    \multicolumn{1}{|l}{Sviluppo del codice della GUI} & 15 & 20 \\
    \multicolumn{1}{|l}{Test e debug} & 5 & 2 \\
    \multicolumn{1}{|l}{Stesura della relazione} & 5 & 3 \\
    \hline
    \textbf{Totale} & \textbf{50} & \textbf{55} \\
    \hline
\end{tabular}
\end{center}
Ci ho messo più della previsione nella progettazione, poiché avevo un'idea
iniziale diversa, che si è pure propagata fino allo sviluppo del modello
(anch'essa conclusa oltre alle ore previste),
tuttavia, dopo alcune considerazioni, sono arrivato alla conclusione che non
avevo abbastanza conoscenze base per l'idea precedente, quindi ho preso invece
un tema a me più vicino. Lo studio di Qt non ha richiesto troppe attenzioni
personalmente, tuttavia essendo questo il mio primo progetto solitario, in un
linguaggio a cui non sono completamente abituato, ha allungato le tempistiche
per lo sviluppo della GUI.

\end{document}