\documentclass[10pt]{article}

\title{StressView}
\author{Alessio Turetta, 2008069}
\date{}

\renewcommand{\contentsname}{Indice}

\begin{document}

\maketitle
\tableofcontents
\newpage

\section{Introduzione}
StressView è un programma di simulazione di performance su un generico componente
per PC.
I tipi di componente a disposizione sono tre dei più importanti, ascelta tra
Processore (CPU), Schede Grafica (GPU) e Memoria (RAM).
Oltre ad offrire un grafico di andamento della performance, il programma segnala
potenziali (soprattutto gravi) perdite rispetto all'andamento normale,
le quali si presentano diversamente a seconda del componente,
e sulla base della temperatura massima inserita da utente (la quale fluttua
nella simulazione)

\section{Modello logico}
\section{Polimorfismo}
\section{Persistenza dei dati}
\section{Funzionalità}
\section{Riepilogo ore}
\end{document}